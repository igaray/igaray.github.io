\addcontentsline{toc}{chapter}{Notes}
\chapter*{Notes}

\begin{itemize}[noitemsep]
\item 
	Rule of Sum
	\begin{itemize}[noitemsep]
	\item How not to use the rule of sum
	\item Sets, Venn diagrams
	\end{itemize}

\item
	Recursive counting

\item
	Tuples and permutations
\end{itemize}

\comment{
Elementos de Algebra
- Calculo combinatorio
  - Principio de multiplicacion
  - Factoriales
  - Variaciones
  - Teorema
  - Demostracion
  - Ejemplos
  - Variaciones con repeticion
  - Teorema
  - Demostracion
  - Ejemplos
  - Combinaciones
  - Teorema
  - Demostracion
  - Ejemplos
  - Combinaciones con repeticion
  - Teorema
  - Demostracion
  - Permutaciones con repeticion
  - Ejemplos
- El desarrollo binomial
  - Teorema
  - Demostracion
  - El Triangulo de Pascal
  - Ejemplos
- Las permutaciones como transformaciones
  - Definicion
  - Ejemplos
  - Teorema
  - Demostracion
  - Ejemplos
  - Definicion
  - Ejemplos
  - Permutaciones pares e impares
  - Ejemplos
  - Definicion
  - Teorema
  - Demostracion
  - Ejemplos
- Ejercicios

Libro: Numeros combinatorios y probabilidades
- 1 La certeza
- 2 Los conjuntos
- 3 Experimentos aleatorios
- 4 Decidir y estimar
- 5 La Ley Clasica de los Grandes Numeros
- 6 Combinatoria: una tecnica para ayudar a contar
  - 1 Comentarios previos
  - 2 Principio basico de conteo.
      Factorial de n
  - 3 Permutaciones sin repeticion
  - 4 Permutaciones con repeticion
  - 5 Los numeros combinatorios
  - 6 El triangulo de Tartaglia o Pascal
  - 7 El simbolo de suma \Sigma
  - 8 Comentarios finales
  - 9 Problemas
- 7 Combinatoria y probabilidad
  - 1 Comentarios previos
  - 2 Mas sobre el factorial n! y el numero combinatorio (n k)
  - 3 El calculo aproximado de n!
  - 4 Algunas aplicaciones y ejemplos
  - 5 Problemas
- 8 Algunos conceptos de Estadistica
  - 1 Comentarios previos
  - 2 Estadisticos de una sucesion de datos numericos
  - 3 Acerca del promedio
  - 4 Acerca del rango
  - 5 Acerca de la mediana
  - 6 Acerca de los fractiles
  - 7 Acerca de la dispersion en una lista de datos
      El rango y su insuficiencia
  - 8 Acerca del analisis y distribucion de los datos: el histograma
  - 9 Estimando la cantidad de taxis que circulan por una gran ciudad
  - 10 Un modelo de simulacion para la experiencia anterior
  - 11 Problemas
- 9 El pasado, el presente, y el futuro

Libro: El Arte de Contar
- 1 Contemos
- 2 Grafos y mapas
- 3 El eterno nomade
- 4 Contando (sin usar los dedos)
- 5 La combinatoria de las cifras
- Anexo: Demostracion del lema de Sperner

Apuntes: Modelos Estadisticos para Ciencias de la Computacion
- 01 Probabilidad
  Introduccion
  Conceptos Estadisticos Basicos
    Experimento
    Espacio muestral
    Algebra de eventos
    Nocion de probabilidad
      Enfoque empirico o frecuentista de probabilidad
      Enfoque clasico de probabilidad
      Axiomas de probabilidad
      Consecuencias de los axiomas
      Asignacion de Probabilidades
    Probabilidad condicional
    Regla de la multiplicacion
    Eventos independientes
    Regla de la multiplicacion para eventos independientes
    Teorema de la probabilidad total
    Teorema de Bayes

- 02 Software Estadistico

- 03 Variables Aleatorias Discretas
  Definicion variable aleatoria
  Definicion variable aleatoria discreta
  Definicion funcion de probabilidad puntual o de masa
  Definicion funcion de distribucion acumulada (fda)
  Propiedades de la funcion de distribucion acumulada
  Esperanza o valor esperado de una variable aleatoria discreta
  Interpretacion de la esperanza
  Propiedades de la esperanza
  Varianza de una variable aleatoria discreta
  Desvio estandar
  Propiedades de la varianza

- 04 Variables Aleatorias Continuas
  Definicion variabla aleatoria continua
  Funcion de densidad de probabilidad
  Funcion de densidad acumulada de una variable aleatoria continua
  Variable aleatoria uniforme
  Variable aleatoria exponencial
  Variable aleatoria gamma
  Esperanza o valor esperado de una variable aleatoria continua
  Propiedad de linealidad
  Varianza de una variabla aleatoria continua
  Distribucion Normal
  Caracteristicas de la funcion de densidad
  Variable aleatoria normal estandar

- 05 Transformacion de Variables Aleatorias
  Funcion de una variable aleatoria discreta
  Funcion de una variable aleatoria continua
  Teorema

- 06 Muestras Aleatorias y Distribucion de Muestreo
  Definicion intuitiva
  Definicion formal
  Definicion estadistico
  Media muestral
  Proposicion
  Teorema Central del Limite
  Varianza muestral
  Distribuciones continuas utilizadas en inferencia estadistica
  Distribucion Chi-Cuadrado
  Distribucion t de Student
- 07 Estimacion de Parametros
- 08 Prueba de Hipotesis
- 09 Analisis de Correlacion
- 10 Analisis de Regresion Lineal
- 11 Teoria de Colas
}